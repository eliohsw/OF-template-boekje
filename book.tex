\documentclass[11pt, a4paper]{memoir}
\message{2026/01/19 OF-template-boekje}

\usepackage[
  inner=2.8cm, outer=2.2cm, top=2.5cm, bottom=2cm,
  showframe % show page layout for debugging
]{geometry}

\makeatletter
\def\input@path{{packages/}}
\makeatother

% Main packages for memoir class
% Options: en (English), de (German), tw (Traditional Chinese), cn (Simplified Chinese),
%          glo (glossary), idx (index)
\usepackage[en]{boekje}

% If Chinese names and labels are needed, load style package with appropriate option
% Options: tw (Traditional Chinese), cn (Simplified Chinese)
% \usepackage[tw]{boekje-label}

% Set code listing style (UNCOMMENT if needed)
% Options: [style=...]
% 1. Pygments styles (See https://pygments.org/styles/ and https://pygments.org/demo/)
% 2. User defined styles: style1, style2, style3, ... (See script/mint_styles.py)
%    Since Overleaf runs a restricted -shell-escape, to use custom styles,
%    compile the _minted/style*.style.minted file locally by `make`, then sync via Git
% \usepackage[style=style1]{boekje-code}


% Set latin script and CJK fonts
% Provided by \usepackage{boekje-fonts}
% Font setup
% Modify this file to change fonts used in the document
% Use fonts installed on your system
% See https://ctan.org/pkg/fontspec and https://ctan.org/pkg/xecjk
% See also packages/set-fonts.sty

% Enable font debugging information
\SetFontsDebugOn

% Usage:
% \SetFontFromList[∅|CJK]{main|sans|mono}{<font1[per-font options], font2[per-font options], ...>}
%
% For each candidate font in the list, the first one that exists is selected.
% If none exist, the corresponding default is restored.

\SetFontFromList{main}{
  CMU Serif, Times New Roman,
  TeX Gyre Termes, Latin Modern Roman
}
\SetFontFromList{sans}{
  Source Sans 3[
    UprightFont    = SourceSans3-Regular,
    ItalicFont     = SourceSans3-Italic,
    BoldFont       = SourceSans3-Bold,
    BoldItalicFont = SourceSans3-BoldItalic,
    FontFace = {sb}{n}{SourceSans3-SemiBold},
    FontFace = {sb}{it}{SourceSans3-SemiBoldItalic},
    FontFace = {sbx}{n}{SourceSans3-SemiBold},
    FontFace = {sbx}{it}{SourceSans3-SemiBoldItalic},
  ], Arial, Fira Sans,
  TeX Gyre Heros, Latin Modern Sans
}
\SetFontFromList{mono}{
  CMU Typewriter Text,
  Source Code Pro, 
  Latin Modern Mono
}

\SetFontFromList[CJK]{main}{
  % Kaiti
  TW-MOE-Std-Kai[
    AutoFakeSlant=0.2,
    AutoFakeBold=3.05
  ], TW-Kai[
    AutoFakeSlant=0.2,
    AutoFakeBold=3.05
  ],
  % Mingti/Songti
  Source Han Serif TC, Noto Serif CJK TC,
  Source Han Serif SC, Noto Serif CJK SC, FandolSong
}

\SetFontFromList[CJK]{sans}{
  Source Han Sans TC[
    UprightFont    = SourceHanSansTC-Regular,
    ItalicFont     = SourceHanSansTC-Regular,
    BoldFont       = SourceHanSansTC-Bold,
    BoldItalicFont = SourceHanSansTC-Bold,
    FontFace = {sb}{n}{SourceHanSansTC-Medium},
    FontFace = {sb}{it}{SourceHanSansTC-Medium},
    FontFace = {sbx}{n}{SourceHanSansTC-Medium},
    FontFace = {sbx}{it}{SourceHanSansTC-Medium},
  ], Noto Sans CJK TC,
  Source Han Sans SC, Noto Sans CJK SC, FandolHei
}
\SetFontFromList[CJK]{mono}{
  Source Han Sans TC, Noto Sans CJK TC,
  Source Han Sans SC, Noto Sans CJK SC, FandolFang
}


% Set language-specific fonts (Polyglossia) (UNCOMMENT if needed)
% % Set language-specific fonts (Polyglossia)

% Usage:
% 1.
% \text<language>{...}
% 2.
% \begin{<language>}
%   Text in the specified language.
% \end{<language>}

% 1. Set Latin based languages
%    Use \set<language>font , \set<language>fontsf , \set<language>fonttt

\setotherlanguage[variant=german, spelling=new]{german}

% 2. Define fonts for other languages
%    Use \newfontfamily
%    Use \<language>font , \<language>fontsf , \<language>fonttt
%    Make sure the font supports the script and language

\setotherlanguage{russian}
\newfontfamily\russianfont[
  Script=Cyrillic,
  Language=Russian
]{Times New Roman}

\setotherlanguage[numerals=greek]{greek}
\newfontfamily\greekfont[
  Script=Greek,
  Language=Greek,
]{Times New Roman}

\setotherlanguage{thai}
\newfontfamily\thaifont[
  Script=Thai,
  Language=Thai,
]{Sarabun}

\setotherlanguage{arabic}
\AtBeginEnvironment{Arabic}{\setlength{\parindent}{0pt}}
\newfontfamily\arabicfont[
  Script=Arabic,
  Language=Arabic,
  Scale=1.0
]{Amiri}
% {arabic} is defined internally by LaTeX, use the environment {Arabic} instead.

% Language mappings
\DeclareLanguageMapping{chinese}{english}
\DeclareLanguageMapping{arabic}{english}
\DeclareLanguageMapping{thai}{english}


% Set document style settings
% Memoir native style settings
\makechapterstyle{MyChapter}{%
  \renewcommand*\chapnamefont{\sffamily\huge\sbseries}%
  \renewcommand*\chapnumfont {\sffamily\huge\sbseries}%
  \renewcommand*\chaptitlefont{\sffamily\Huge\sbseries}%
  \renewcommand*\printchaptername{\chapnamefont \MakeUppercase{\chaptername}}%
  \renewcommand*\printchapternum{\chapnumfont\ \thechapter}%
  \renewcommand*\afterchapternum{\par\vskip 10pt}%
}
\chapterstyle{MyChapter}

\setsecheadstyle{\sffamily\Large\sbseries}
\setsubsecheadstyle{\sffamily\large\sbseries}
\setsubsubsecheadstyle{\sffamily\normalsize\sbseries}

% Enable indexing (UNCOMMENT if needed)
% Use \index{<entry>} or \idx{<entry>} to add index entries
% \index{<entry>!<subentry>}
% \index{<entry>@<sort key>}
% \index{<entry>|(} ... \index{<entry>|)}
% \index{<entry>|textbf}
% \newcommand{\idx}[1]{\index{#1}#1}
% \makeindex

% Enable acronyms and glossary (UNCOMMENT if needed)
% % Insert this file in the preamble
% See also: https://overleaf.com/learn/latex/Glossaries

% Usage: \newacronym{<label>}{<short>}{<long>}
%        \newglossaryentry{<label>}{name=<name>, description={...}}
% E.g.,  \newacronym{adc}{ADC}{analog-to-digital converter}
%        \newglossaryentry{meter}{name=meter, description={Some description}}

% Useful keys: `firstplural`, `plural`, `longplural`, `description`
% \newacronym[
%   plural={ADCs},
%   firstplural={analog-to-digital converters (ADCs)},
%   longplural={analog-to-digital converters},
%   description={Some description}
% ]{adc}{ADC}{analog-to-digital converter}

% Refer to acronyms and glossary entries with 
%   \gls{},\Gls{},\glspl{},\Glspl{} 
% Forcing a specific form of acronyms with
%   \acrshort{},\acrlong{},\acrfull{},\acrshortpl{},\acrlongpl{},\acrfullpl{}

% Set acronym as used or unused with \glsunset{} \glsreset{}
% Reset all acronyms with \glsresetall{}

% --------------------------------------------------------------------------



% --------------------------------------------------------------------------



% \makeglossaries

% Enable bibliography (UNCOMMENT if needed)
% \addbibresource{contents/bibliography/references.bib}

\raggedbottom

% ================================================================================================

\begin{document}

\justifying
\setlength{\parindent}{22pt}

% Title settings

\makeatletter
\pretitle{%
  \vspace*{6cm}
  % \centering
  \fontsize{40}{48}\selectfont\BebasNeue\bfseries
}
\posttitle{%
  \par\vspace{0.1cm}
  \huge\sffamily\sbseries
  \@subtitle\par
  \vspace{2cm}%
}

\newcommand{\subtitle}[1]{\gdef\@subtitle{#1}}
\newcommand{\@subtitle}{}

\preauthor{%
  % \centering
  \Large\sffamily
}
\postauthor{%
  \par\vspace{0.25cm}
  \large{\@idnumber}\par
  \vspace{3.5cm}
}

\newcommand{\idnumber}[1]{\gdef\@idnumber{#1}}
\newcommand{\@idnumber}{}

\predate{%
  % \centering
  \Large
}
\postdate{%
  \par\vspace{0.25cm}
  \Large{\@placename}\par
  \vfill
}

\newcommand{\placename}[1]{\gdef\@placename{#1}}
\newcommand{\@placename}{}
\makeatother

% ---------------------------------------------------------------
% FILL IN TITLE PAGE INFORMATION HERE
\title{Main Title}
\subtitle{Subtitle}
\author{Author Name}
\idnumber{B1234567890}
\date{\DTMdisplaydate{2025}{1}{1}{-1} -- \DTMdisplaydate{2025}{1}{2}{-1}}
\placename{Place Name}
% ---------------------------------------------------------------

% Title page layout
\hypersetup{pageanchor=false}
\pagenumbering{gobble}
\begin{titlingpage}
  % Vertical rule on the left side
  \begin{tikzpicture}[remember picture,overlay]
    \draw[line width=1.5pt]
      ($(current page text area.north west) + (0.5cm, -2cm)$) -- 
      ($(current page text area.south west) + (0.5cm, 2cm)$);
  \end{tikzpicture}

  % Title content
  \noindent
  \hspace*{1.5cm}% Move text right, slightly more than 2cm to clear the rule
  \begin{minipage}{0.8\textwidth}
    \maketitle
  \end{minipage}
\end{titlingpage}

\hypersetup{pageanchor=true}
\pagenumbering{roman}

% Page style settings
\makepagestyle{MyPage}
\makeevenhead{MyPage}{\sffamily\itshape\leftmark}{}{}
\makeoddhead {MyPage}{}{}{\sffamily\rightmark}
\makeevenfoot{MyPage}{}{\thepage}{}
\makeoddfoot {MyPage}{}{\thepage}{}

% Head and foot rules (UNCOMMENT if needed)
% \setlength{\headwidth}{\textwidth}
% \makeheadrule{MyPage}{\headwidth}{0.4pt}
% \makefootrule{MyPage}{\headwidth}{0.4pt}{\footruleskip}

\pagestyle{MyPage}

% ================================================================================================

\frontmatter

% Add preface here (UNCOMMENT if needed)
% \chapter*{Preface}
\addcontentsline{toc}{chapter}{Preface}

This is the preface text. You might explain the purpose and scope here.

% Make sure to add a new page after the preface
\newpage

% Print Table of Contents
\renewcommand{\cftdot}{} % Remove dots in TOC
\renewcommand{\printtoctitle}[1]{\Huge\sffamily\sbseries #1} % Set TOC title font

\tableofcontents

% ================================================================================================

\mainmatter

% Epigraph settings
\setlength{\epigraphrule}{0.0pt}
\setlength{\epigraphwidth}{0.6\textwidth}
% \renewcommand{\textflush}{flushleft}
% \renewcommand{\sourceflush}{flushright}

% ADD MORE chapters here
% Chapters may be separated by parts, use \part{} between chapters.
% \part{Part Name}
\chapter{Chapter Name}

% Add an epigraph (UNCOMMENT if needed)
% \epigraph{\sffamily
%   Pure mathematics is, in its way, the poetry of logical ideas.
% }{\sffamily Albert Einstein}


\begin{codelisting}{python}{}
import numpy as np
from scipy.stats import weibull_min
\end{codelisting}

% Appendices (UNCOMMENT if needed)
% \appendix
% \renewcommand*\printchaptername{\chapnamefont \MakeUppercase{Appendix}}%

% ADD MORE appendices here
% \input{contents/appendices/appendix01.tex}

% ================================================================================================

\backmatter

% Print Acronyms (UNCOMMENT if needed)
% \printglossary[type=\acronymtype]
% \addcontentsline{toc}{chapter}{Acronyms}

% Print Glossary (UNCOMMENT if needed)
% \printglossary
% \addcontentsline{toc}{chapter}{Glossary}

% Print Bibliography (UNCOMMENT if needed)
% \printbibliography[heading=bibintoc]

% Print Index (UNCOMMENT if needed)
% \printindex

\end{document}